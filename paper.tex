\newcount\draft\draft=1  % set to 0 for submission

\documentclass[10pt,preprint,numbers]{sigplanconf}

\input{packages}  % package imports
\input{macros}    % handy macros

\begin{document}

% conference info %%%%%%%%%%%%%%%%%%%%%%%%%%%%%%%%%%%%%%%%%%%%%%%%%%%%%%%%%%%%%%
% \conferenceinfo{ASPLOS~'16}{April 2--6, 2016, Atlanta, Georgia, USA} 
% \copyrightyear{2016} 
% \copyrightdata{978-1-4503-4091-5/16/04} 
% \copyrightdoi{2872362.2872406} 
% \publicationrights{licensed}

% title & authors %%%%%%%%%%%%%%%%%%%%%%%%%%%%%%%%%%%%%%%%%%%%%%%%%%%%%%%%%%%%%%
\title{Make Halting Problem an Unsolved Problem Again by Carefully Designing a Purely Functional Programming Language}

\authorinfo{tsao chi}
           {a certain High School :)}
           {tsao-chi@the-lingo.org}

\maketitle

% abstract & classification %%%%%%%%%%%%%%%%%%%%%%%%%%%%%%%%%%%%%%%%%%%%%%%%%%%%
\begin{abstract}
% PROBLEM: what problem are you addressing?
% CONTRIBUTION: what have you done to solve this problem?
% RESULT: what was the outcome of your contribution to the problem?
% MEANING: what does this outcome mean?

While the halting problem is proven to be undecidable in conventional purely functional programming languages,
it's possible to design a Turing-complete purely functional programming language
by breaking some fundamental properties of conventional purely functional programming languages
such that the problem of implementing an interpreter which will finish running for any input program is unsolved.
\end{abstract}

{\footnotesize
\category{D.3.3}{Programming Languages}{Language Constructs and Features}
%\terms  % terms are optional
%Term One; Term Two; Term Three;
\keywords
Halting Problem; Language Design;
}

% the actual paper! %%%%%%%%%%%%%%%%%%%%%%%%%%%%%%%%%%%%%%%%%%%%%%%%%%%%%%%%%%%%

\section{Introduction}

The key is to cancel the uniqueness property of normal forms, then the proof of the undecidability of whether the process of interpreting an expression will halt, known as Turing's proof~\citep{turing_paper}, is invalid.

\subsection{The Design}

It's a Scheme dialect. Mostly the same as Scheme, except for the following items.

\begin{enumerate}
\item purely functional and lazy
\item the Exception Handling System
\end{enumerate}

\section{Background - How the idea was conceived}

When I was in junior high school, I was playing with my Scheme-like purely functional language and
came up with these 2 features.

\subsection{the Exception Handling System - Exceptions are just ordinary values}

Functions that do not expect an exception as input will generate a new exception which usually (it's not forced by language specialization, so ``usually'') includes the original exception when the input is an exception.

\subsection{the cancellation of the uniqueness property of normal forms}

I observed that there are some expressions which have not naturally unique normal form, such as converting a finite set into a list.

\section{the Rule}

With these 2 features,
it's reasonable to allow an interpreter to interpret an expression that will run forever as an exception.

\section{Effects of the Rule}

\subsection{Non-uniqueness caused by the Rule}

Consider this example. Interpret {\scriptsize\verb|a|} or {\scriptsize\verb|b|} or the entire expression as an exception is all reasonable.
\lstset{language=Lisp}
\begin{lstlisting}[frame=single]
(letrec ([a (car b)] [b (car a)]) a)
\end{lstlisting}

\subsection{Range}

Interpreting the entire program as an Exception conforms to the rule, but this bypasses the program's exception handling system. So an interpreter should try to avoid this situation.

\section{How is this language Turing-complete?}

A computational system that can compute every Turing-computable function is called Turing-complete (or Turing-powerful).~\citep{wikipedia_Turing_completeness}

The Rule does nothing to Turing-computable function, so it's Turing-complete.

\section{Further work}

\begin{enumerate}
\item Resolve the problem.
\item Formalize this language.
\end{enumerate}

% bibliography %%%%%%%%%%%%%%%%%%%%%%%%%%%%%%%%%%%%%%%%%%%%%%%%%%%%%%%%%%%%%%%%%
\bibliographystyle{abbrvnat}
\bibliography{paper}

\end{document}
